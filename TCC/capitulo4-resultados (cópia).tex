\chapter[Experimentos e Resultados]{Experimentos e Resultados}

O método desenvolvido foi avaliado de duas formas: a primeira com experimentos sobre fotos de tabuleiros de Go com diferentes configurações de pedras e a segunda com partidas reais.

\section{Testes com fotos}

A primeira avaliação do sistema foi feita aplicando-se as duas primeiras etapas do método (detecção de tabuleiro e detecção de pedras) sobre fotos de tabuleiros de Go com diferentes configurações de pedras. Os objetivos deste experimento foram medir a precisão da parte de visão computacional do sistema e fazer o ajuste dos parâmetros dos algoritmos utilizados.

As fotos foram tomadas com diferentes características de fonte de luz, quantidade de pedras no tabuleiro e presença de sombras na imagem, de forma a simular as diferentes condições visuais que podem ocorrer durante uma partida real.

Dois exemplos de fotos utilizadas no experimento são apresentadas nas Figuras~\ref{fig:experimento1-foto-exemplo1} e~\ref{fig:experimento1-foto-exemplo2}, juntamente com suas características.

\begin{figure}[tbh]
\includegraphics[width=10cm]{imagens/experimento1_foto_exemplo1.jpg}
\centering
\caption[Exemplo de foto utilizada no primeiro experimento]{Exemplo de foto utilizada no primeiro experimento. Características: Fonte de luz branca, duas pedras, há sombras.}
\label{fig:experimento1-foto-exemplo1}
\end{figure}

\begin{figure}[tbh]
\includegraphics[width=10cm]{imagens/experimento1_foto_exemplo2.jpg}
\centering
\caption[Outro exemplo de foto utilizada no primeiro experimento]{Outro exemplo de foto utilizada no primeiro experimento. Características: Fonte de luz branca, onze pedras, sem sombras.}
\label{fig:experimento1-foto-exemplo2}
\end{figure}

%A seguir, a Tabela~\ref{tab:resumo-experimento1} traz um resumo de todas as fotos utilizadas neste experimento juntamente com suas características e o resultado obtido pelo sistema, ou seja, se o tabuleiro e as pedras foram identificados corretamente na imagem.

% By http://www.tablesgenerator.com/
%\begin{table}[tbh]
%\centering
%\caption{Resumo dos resultados do experimento 1}
%\label{tab:resumo-experimento1}
%\begin{tabular}{l|l|l|l|l}
%\textbf{Foto} & \textbf{Fonte de luz} & \textbf{N\textordmasculine  de pedras} & \textbf{Presença de sombras} & \textbf{Resultado}\\
%\hline
%1    & Branca       & 0                & Não                 & Sim              \\
%\hline
%2    & Branca       & 0                & Sim                 & Sim              \\
%\hline
%3    & Branca       & 0                & Sim                 & Sim          
%\end{tabular}
%\end{table}

Neste teste foi utilizado um conjunto de 66 fotos de tabuleiros. Desse total, 58 fotos tiveram suas pedras detectadas corretamente, resultando em uma precisão de 87,9\%. 
%Conforme apresentado na tabela, X\% das fotos tiveram seus tabuleiros detectados corretamente.
Pode-se observar pelos resultados que um dos principais obstáculos para a correta identificação das imagens é a presença de luz em excesso ou sombras sobre o tabuleiro, que gera resultados incorretos. O problema das sombras é que, quando são muito escuras, se confundem com possíveis pedras pretas. Já quando há excesso de luz ocorrem pontos de maior luminosidade no tabuleiro, levando o detector a encontrar pedras brancas onde não existem.

Verificou-se também que, quando uma foto com muitas pedras é apresentada ao sistema, especialmente se estiverem sobre as bordas, o tabuleiro não é encontrado pois as peças encobrem boa parte das suas linhas - um dos problemas mencionados por~\citeonline{hirsimaki05}. Entretanto, como já mencionado no capítulo anterior, o método proposto neste trabalho separa as etapas de detecção do tabuleiro e de detecção das pedras e, além disso, impõe as restrições de que o tabuleiro deve começar vazio e permanecer imóvel. Dessa forma, esse problema ocorre apenas na detecção de pedras em imagens estáticas, já que no registro de uma partida o tabuleiro é detectado antes de as pedras começarem a ser colocadas.

\section{Experimento com partidas reais}

Considerando que nas fotos o resultado da etapa de visão computacional do método foi bom, indicando um desempenho satisfatório do sistema, partiu-se para os testes com registro de partidas reais. Sete partidas entre jogadores do Clube de Go de São Carlos~\footnote{\url{http://www.vamosjogargo.com/}} foram registradas. Um tripé com suporte para smartphone foi utilizado para manter a câmera fixa durante o registro das partidas.

Todas as sete partidas foram registradas corretamente. Contudo, durante o processo de registro ocorreram alguns problemas: o primeiro foi que quando pedras de material reflexivo (plástico, por exemplo) eram utilizadas e havia uma fonte de luz incidindo diretamente sobre elas, as pedras pretas eram confundidas com pedras brancas pelo sistema devido ao reflexo. A solução para esse problema foi utilizar pedras foscas, que produzem pouca reflexão.

As distorções causadas pela visualização do tabuleiro em perspectiva foram a origem de outro problema. A borda do tabuleiro mais distante da câmera, por sofrer maior distorção, ocasionalmente sofria interferência dos objetos adjacentes ao tabuleiro. Neste caso, a solução foi tentar manter a câmera com um ângulo de visão menos agudo em relação ao tabuleiro, o que diminui bastante esse problema. \mudar{Uma angulação de ao menos 45 graus (considerando uma visão ortogonal em 90 graus) já foi suficiente.}

Movimentos da câmera ou do tabuleiro, mesmo que pequenos, atrapalham a detecção, devendo ser evitados. Um último problema encontrado foi a propensão do detector a encontrar pedras brancas nas posições de borda do tabuleiro. Isso ocorreu principalmente no registro de partidas durante o entardecer, com parte da iluminação proveniente da luz solar. Esse tipo de luz tem maior intensidade que a luz branca artificial, o que leva o detector por vezes a confundir algumas interseções vazias com pedras brancas, conforme havia sido discutido anteriormente.

Os falsos positivos encontrados pelo sistema foram tratados de forma manual, já que o aplicativo provê a funcionalidade de voltar a última jogada registrada para os casos de registro incorreto. Dessa forma, quando uma jogada incorreta era detectada, ela era desfeita e tentava-se registrar a jogada correta novamente. Não houveram falsos negativos já que todas as jogadas corretas das partidas foram detectadas. A Tabela~\ref{tab:falsos-positivos} a seguir mostra o número de falsos positivos e a precisão\footnote{A precisão é calculada como o número de jogadas detectadas corretamente dividido pelo número total de jogadas detectadas.} encontrados durante o registro de cada partida.

\begin{table}[tbh]
\centering
\label{tab:falsos-positivos}
\begin{tabular}{l|l|l|l}
\textbf{Partida} & \textbf{Número de jogadas} & \textbf{Falsos positivos} & \textbf{Precisão}\\
\hline
1 &  37 &  0 &   100\%\\    % Leo x Grace   2016-01-12
\hline
2 &  48 &  1 & 97,96\%\\    % Teste         2016-03-26 1211
\hline
3 &  54 &  0 &   100\%\\    % Grace x Alceu 2016-04-02
\hline
4 & 244 & 11 & 95,69\%\\    % Leo x Pinguim 2016-03-29
\hline
5 & 254 & 10 & 96,21\%\\    % Leo x Daniel  2016-04-02
\hline
6 & 267 &  0 &   100\%\\    % Leo x Renata  2016-04-03
\hline
7 & 237 &  2 & 99,16\%\\    % Leo x Sato    2016-05-15
\hline
\textbf{Média} & 163 & 3,43 & 98,43\%\\
\end{tabular}
\caption{Número de falsos positivos e precisão encontrados em cada registro}
\end{table}

Apesar da inconveniência de corrigir manualmente os falsos positivos, o registro das partidas não foi prejudicado.
%correu sem grandes transtornos. % Um ponto positivo foi que nenhuma pedra foi detectada em posição errada, indicando que a detecção da posição das pedras está bem robusta.
Os registros dessas partidas são apresentados nas Figuras~\ref{fig:partida1} a~\ref{fig:partida6} a seguir.

\begin{figure}[htb!]
\centering
\begin{minipage}{.49\textwidth}
  \includegraphics[width=7.5cm]{imagens/partida_registrada1.jpg}
  \centering
  \caption{Primeira partida registrada.}
  \label{fig:partida1}
\end{minipage}
\begin{minipage}{.5\textwidth}
  \includegraphics[width=7.5cm]{imagens/partida_registrada2.jpg}
  \centering
  \caption{Segunda partida registrada.}
  \label{fig:partida2}
\end{minipage}
\end{figure}

\begin{figure}[htb!]
\centering
\begin{minipage}{.49\textwidth}
  \includegraphics[width=7.5cm]{imagens/partida_registrada3.png}
  \centering
  \caption{Terceira partida registrada.}
  \label{fig:partida3}
\end{minipage}
\begin{minipage}{.5\textwidth}
  \includegraphics[width=7.5cm]{imagens/partida_registrada4.png}
  \centering
  \caption{Quarta partida registrada.}
  \label{fig:partida4}
\end{minipage}
\end{figure}

\begin{figure}[htb!]
\centering
\begin{minipage}{.49\textwidth}
  \includegraphics[width=7.5cm]{imagens/partida_registrada5.png}
  \centering
  \caption{Quinta partida registrada.}
  \label{fig:partida5}
\end{minipage}
\begin{minipage}{.5\textwidth}
  \includegraphics[width=7.5cm]{imagens/partida_registrada6.png}
  \centering
  \caption{Sexta partida registrada.}
  \label{fig:partida6}
\end{minipage}
\end{figure}

\begin{figure}[htb!]
\centering
\includegraphics[width=7.5cm]{imagens/partida_registrada7.png}
\centering
\caption{Sétima partida registrada.}
\label{fig:partida7}
\end{figure}

\section{Discussão}

Os testes com fotos apresentaram alguns problemas, particularmente em relação à presença de sombras nas imagens e ao tipo de fonte de luz. Contudo, o registro de partidas se mostrou bastante robusto. Possibilidades de melhoria para as dificuldades encontradas são discutidas no capítulo seguinte.

Eventuais erros na etapa de detecção do tabuleiro são mitigados pois o usuário deve confirmar a posição do tabuleiro detectado antes de iniciar o registro da partida. Dessa forma, mesmo que a detecção inicial tenha sido incorreta, o usuário pode mudar a posição da câmera até que o sistema detecte a posição do tabuleiro corretamente.

A etapa de detecção de pedras foi a que mais ocasionou erros, na forma de falsos positivos, especialmente em relação às pedras brancas. A detecção de pedras pretas é mais simples pois o contraste delas com o tabuleiro é alto, independentemente da iluminação. As brancas, contudo, são menos distinguíveis, dependendo das condições de luz do ambiente. Muitas variações na luminosidade sobre o tabuleiro prejudicam a correta detecção das pedras, já que esse processo se baseia em cálculos de médias de cores. %Esses problemas levaram à utilização de certos tipos de tabuleiro e pedras na realização dos testes.

Por sua vez, a detecção de jogadas - considerando que a detecção das pedras tenha sido feita corretamente - ocorreu sem problemas. A restrição de dois segundos para detectar uma jogada poderia ser diminuída para um segundo e meio ou um segundo, mas testes adicionais são necessários para determinar o tempo ideal.

Também é importante ressaltar que diferentes smartphones e tablets podem trazer resultados distintos dos apresentados devido à diversidade de resoluções e qualidades de câmera.

Um último ponto de importância é que o aplicativo faz uso intensivo do processador do smartphone, consumindo bastante bateria e aumentando a temperatura do dispositivo. Melhoramentos no sentido de diminuir o processamento para aumentar a autonomia do aplicativo são possibilidades de trabalhos futuros.

\subsection{Comparação com trabalhos relacionados}

O presente trabalho é o primeiro a fazer o registro de partidas de Go utilizando exclusivamente dispositivos móveis e um dos poucos que realiza o registro completo de uma partida. Além disso, os métodos de avaliação e os dados utilizados nos experimentos são distintos dos demais. %Particularmente neste trabalho, quando algum problema era encontrado (pedra não detectada ou detectada erroneamente) os usuários removiam o erro manualmente do registro e o sistema tentava detectar novamente a jogada correta. Esse processo era repetido até que a jogada fosse detectada corretamente.
Essas diferenças dificultam a comparação deste trabalho com as pesquisas relacionadas citadas.

Contudo, algumas comparações gerais são possíveis. Comparado a~\citeonline{yanai06}, que registra jogos a partir de transmissões de partidas pela televisão, o método proposto neste trabalho é mais versátil e prático, já que a câmera não precisa estar localizada exatamente acima do tabuleiro.% \citeonline{srisuphab12} também compartilham dessa limitação de ângulo de visão do tabuleiro.}

O trabalho atual combina características das pesquisas de~\citeonline{hirsimaki05} (transformada de Hough, média de cores ao redor das interseções), \citeonline{scher08} (transformação ortogonal do tabuleiro), \citeonline{srisuphab12} (biblioteca OpenCV, câmera fixa), \citeonline{musil14} e~\citeonline{carta15}. Uma comparação geral do desempenho dos trabalhos relacionados com a do sistema desenvolvido neste trabalho é apresentada na Tabela~\ref{tab:comparacao-precisao} a seguir.

\begin{table}[htb]
    \label{tab:comparacao-precisao}
    
    \begin{footnotesize}
    \begin{center}
    \begin{tabular}{p{4.6cm}|p{3.2cm}|p{1.6cm}|p{5cm}}
    
    \textbf{Trabalho} & \textbf{Aplicado sobre} & \textbf{Precisão} & \textbf{Precisão calculada sobre}\\
    \hline
    \cite{yanai06} & Jogos televisionados & 95,7\% & Jogadas detectadas corretamente\\
    \hline
    \cite{hirsimaki05} & Fotos & -- & --\\
    \hline
    \cite{scher08} & Sequências de fotos & 91,34\% & Jogadas detectadas corretamente em jogos com poucas jogadas\\
    \hline
    \cite{seewald10} & Fotos & 72,7\% & Imagens com todas as pedras identificadas corretamente\\
    \hline
    \cite{srisuphab12} & Partidas ao vivo & -- & --\\
    \hline
    \cite{musil14} & Fotos & 76,4\% & Imagens com todas as pedras identificadas corretamente\\
    \hline
    \cite{carta15} & Sequências de fotos & 95\%-100\% & Jogadas detectadas corretamente em condições ideais\\
    \hline
    \hline
    \textbf{Kifu Recorder} & Partidas ao vivo & 98,43\% & Jogadas detectadas corretamente em boas condições de iluminação\\
    \end{tabular}
    \end{center}
    \end{footnotesize}
    \caption{Comparação da precisão dos resultados}
\end{table}

Ao analisar os dados da tabela é importante levar em consideração que foram utilizados diferentes métodos de avaliação e dados em cada trabalho. Além disso, o Kifu Recorder impõe algumas restrições para seu funcionamento que tornam a tarefa de anotação de partidas um pouco mais simples. Porém, em linhas gerais, os resultados apresentados pelo aplicativo são expressivos e comparáveis ao estado da arte~\cite{carta15}, considerando que os registros foram completados com sucesso e foram feitos sobre partidas reais, com tabuleiros de tamanho normal (19x19) e utilizando exclusivamente dispositivos móveis.

O Kifu Recorder não pôde ser comparado aos aplicativos pesquisados na Seção~\ref{sec:aplicativos-para-dispositivos-moveis} pois eles não disponibilizam dados de precisão. Entretanto, é importante mencionar que o Kifu Recorder não necessita de uma etapa de calibração manual da posição do tabuleiro (obrigatória no KifuSnap e GoEye) e nem de calibração de cores, (necessária no Baduk Cap e PhotoKifu).
